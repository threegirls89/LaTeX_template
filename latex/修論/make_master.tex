\documentclass[fleqn]{jreport}%<--- class はjreport
%package%{{{
\usepackage{sotsuron_14}         %<----- 修論、卒論用のスタイルファイル

% いつもの奴ら。 ここもpreamble.texに書こう
\usepackage{booktabs}
\usepackage[dvipdfmx]{graphicx,color}
%\usepackage[dvipdfmx]{color}
\usepackage{amsmath}
\usepackage{ascmac}
\usepackage{times}
\usepackage{epsf}
\usepackage{verbatim}
\usepackage{subfigure}
\usepackage{multicol}
\usepackage{pifont}
\usepackage{mathtools} % dcases を使う

%ハイパーリンク  以下も参照:http://ossyaritoori.hatenablog.com/entry/2016/11/11/%E3%80%90Tex%E3%80%91%E5%8D%92%E8%AB%96%E4%BF%AE%E8%AB%96%E3%81%AB%E3%81%AF%E6%98%AF%E9%9D%9E%E3%83%8F%E3%82%A4%E3%83%91%E3%83%BC%E3%83%AA%E3%83%B3%E3%82%AF%E6%A9%9F%E8%83%BD%E3%82%92%E3%81%A4
\usepackage[dvipdfmx, bookmarkstype=toc, colorlinks=false, pdfborder={0 0 0}, bookmarks=true, bookmarksnumbered=true]{hyperref}
\usepackage{pxjahyper}


% 修論用
\usepackage{docmute} %単体でコンパイルできるようにする

\renewcommand{\bf}{\bfseries}
\renewcommand{\gt}{\gtfamily}
\renewcommand{\sf}{\sffamily}

% algorithmicのstyファイルはデフォルトで存在しないので使わないならコメントアウトしてよし
\label{alg1}   
\usepackage{algorithm}
\usepackage{algorithmic}
%\newcommand{\AND}{\algorithmicand{} }
\renewcommand{\algorithmicrequire}{\textbf{Input:}}
\renewcommand{\algorithmicensure}{\textbf{Output:}}

%諸々の定義 -> preamble.texに書く。

\newcommand{\bm}[1]{\mbox{\boldmath $#1$}}
\newcommand{\eq}[1]{(\ref{eq:#1})}
\newcommand{\zu}[1]{\fig{#1}}
\newcommand{\shiki}[1]{式(\ref{eq:#1})}
\newcommand{\fig}[1]{Fig.\ \ref{fig:#1}}
\newcommand{\grp}[1]{Fig.\ \ref{grp:#1}}
\newcommand{\tb}[1]{Tab.\ \ref{tb:#1}}
\newcommand{\defeq}{:=}
\newcommand{\bvec}[1]{\mbox{\boldmath $#1$}} %太字
\newcommand{\tvec}[1]{\bvec{#1}^{\mathsf{T}}} %太字転置

\newtheorem{theorem}{Theorem}
\newtheorem{definition}[theorem]{Definition}
\newtheorem{lemma}[theorem]{Lemma}
\newtheorem{corollary}[theorem]{Corollary}


\renewcommand{\figurename}{Fig.}
\renewcommand{\tablename}{Tab.}
\graphicspath{{./fig/}} %図はきちんと別のファイルにおいておこう。
\usepackage{url}

\def\vector#1{\mbox{\boldmath $#1$}}

%%%%%%%%%%%%%%%%%  表紙 %%%%%%%%%%%%%%%%%%%
\thesis{
	{\large
		東京大学 \ 大学院新領域創成科学研究科\\
		基盤科学研究系\\
		先端エネルギー工学専攻\\
	}
	\vskip 3em
	平成28年度\\
	\vskip 1em
	修士論文 \\
}
\title{\LARGE
	イメージヤコビアンの特性に着目した特徴量の選択に基づく
	イメージベースビジュアルサーボの性能向上とその応用に関する研究
	\smallskip
	\large\bf
}
\date{\large
	2017年1月30日提出 \\
}
\vskip -3em
\professor{\large
	藤本 博志 准教授\\
}

\author{\Large
%	47-156090
 李 尭希
}



%$$$$$$$$$$$$$$$$$$$$$$$$$$$$$ YNU report paper size $$$$$$$$$$$$$$$$$$$$$$
\topmargin -15mm
\oddsidemargin -0.1in \evensidemargin -0.1in
\baselineskip7.833mm
\textheight 25.2cm
\textwidth 17cm




\begin{document}
\maketitle
\maegaki
\include{chapter/00_Abstract4} %見ての通りアブスト
%
%
\maetsuke
\tableofcontents
\listoffigures
\listoftables

%%%%%%%%%%%%%%%%%%%%     このあたりから本文    %%%%%%%%%%%%%%%%%%%%%%%%%%%%%%%%%%%%%%%%%%%%%
\hombun
\documentclass[fleqn]{jreport}
\usepackage[dvipdfmx, bookmarkstype=toc, colorlinks=false, pdfborder={0 0 0}, bookmarks=true, bookmarksnumbered=true]{hyperref}
\usepackage{pxjahyper}
\usepackage{docmute}

\usepackage{booktabs}
\usepackage[dvipdfmx]{graphicx,color}
%\usepackage[dvipdfmx]{color}
\usepackage{amsmath}
\usepackage{ascmac}
\usepackage{times}

\usepackage{epsf}
\usepackage{verbatim}
\usepackage{subfigure}
\usepackage[varg]{txfonts}

\usepackage{multicol}
\usepackage{pifont}
%\usepackage{docmute} %単体でコンパイルできるようにする
\usepackage{mathtools} % dcases を使う

\renewcommand{\bf}{\bfseries}
\renewcommand{\gt}{\gtfamily}
\renewcommand{\sf}{\sffamily}
\label{alg1}

%\usepackage{algorithm}
%\usepackage{algorithmic}
%\newcommand{\AND}{\algorithmicand{} }
%\renewcommand{\algorithmicrequire}{\textbf{Input:}}
%\renewcommand{\algorithmicensure}{\textbf{Output:}}

\newcommand{\bm}[1]{\mbox{\boldmath $#1$}}
\newcommand{\eq}[1]{(\ref{eq:#1})}
\newcommand{\zu}[1]{\fig{#1}}
\newcommand{\shiki}[1]{式(\ref{eq:#1})}
\newcommand{\fig}[1]{Fig.\ \ref{fig:#1}}
\newcommand{\grp}[1]{Fig.\ \ref{grp:#1}}
\newcommand{\tb}[1]{Tab.\ \ref{tb:#1}}
\newcommand{\defeq}{:=}
\newcommand{\bvec}[1]{\mbox{\boldmath $#1$}} %太字
\newcommand{\tvec}[1]{\bvec{#1}^{\mathsf{T}}} %太字転置

\newtheorem{theorem}{Theorem}
\newtheorem{definition}[theorem]{Definition}
\newtheorem{lemma}[theorem]{Lemma}
\newtheorem{corollary}[theorem]{Corollary}

\graphicspath{{../fig/}}



\begin{document}
\chapter{序論}\label{Introduction}
	
\section{研究背景}
イメージセンサは対象に対して接触せずに非常に多量の情報を取得することができるため,古くから産業用機械や移動ロボット等に環境センサとして多く利用されてきた\cite{Hashimoto2009}\cite{Corke2011}。

\end{document}
%\include{chapter/VisualServo}
\documentclass[fleqn]{jreport}
\usepackage[dvipdfmx, bookmarkstype=toc, colorlinks=false, pdfborder={0 0 0}, bookmarks=true, bookmarksnumbered=true]{hyperref}
\usepackage{pxjahyper}
\usepackage{docmute}

\usepackage{booktabs}
\usepackage[dvipdfmx]{graphicx,color}
%\usepackage[dvipdfmx]{color}
\usepackage{amsmath}
\usepackage{ascmac}
\usepackage{times}

\usepackage{epsf}
\usepackage{verbatim}
\usepackage{subfigure}
\usepackage[varg]{txfonts}

\usepackage{multicol}
\usepackage{pifont}
%\usepackage{docmute} %単体でコンパイルできるようにする
\usepackage{mathtools} % dcases を使う

\renewcommand{\bf}{\bfseries}
\renewcommand{\gt}{\gtfamily}
\renewcommand{\sf}{\sffamily}
\label{alg1}

%\usepackage{algorithm}
%\usepackage{algorithmic}
%\newcommand{\AND}{\algorithmicand{} }
%\renewcommand{\algorithmicrequire}{\textbf{Input:}}
%\renewcommand{\algorithmicensure}{\textbf{Output:}}

\newcommand{\bm}[1]{\mbox{\boldmath $#1$}}
\newcommand{\eq}[1]{(\ref{eq:#1})}
\newcommand{\zu}[1]{\fig{#1}}
\newcommand{\shiki}[1]{式(\ref{eq:#1})}
\newcommand{\fig}[1]{Fig.\ \ref{fig:#1}}
\newcommand{\grp}[1]{Fig.\ \ref{grp:#1}}
\newcommand{\tb}[1]{Tab.\ \ref{tb:#1}}
\newcommand{\defeq}{:=}
\newcommand{\bvec}[1]{\mbox{\boldmath $#1$}} %太字
\newcommand{\tvec}[1]{\bvec{#1}^{\mathsf{T}}} %太字転置

\newtheorem{theorem}{Theorem}
\newtheorem{definition}[theorem]{Definition}
\newtheorem{lemma}[theorem]{Lemma}
\newtheorem{corollary}[theorem]{Corollary}

\graphicspath{{../fig/}}



\begin{document}
\chapter{冗長な特徴点を用いたビジュアルサーボの性能向上}\label{Sencitivity}
本稿では序論で述べた特徴点ベースの手法と領域ベースの手法のうち,特徴点を用いたイメージベースビジュアルサーボの手法について考察を行う。

\section{特徴点の個数に関する考察}
特徴点の個数について,6自由度の制御を行う場合にはイメージヤコビアンのランク落ちを防ぐために最低4つ必要であることがよく知られている\cite{Chaumette2006}。
したがってマーカを用いる場合には4つ以上の点を持つマーカを作成して必要な特徴点を得ていた。

一方で,近年発展しているSIFT\cite{Lowe1999}等の高度な特徴点抽出手法を用いることで画像内から4つ以上の冗長な特徴点を抽出できる。
本章ではこれらの冗長な特徴点を活用することでビジュアルサーボの性能を向上させることを目指す。


\subsection{感度の定義と冗長特徴点による感度の上昇 }
すでに文献\cite{Hashimoto1998}において,冗長な特徴点を制御に用いることによってタスク実現の速度や精度を改善できる事が示されている。

ここでは文献に倣って感度の定義と冗長に特徴点を取ることにより感度が上昇することを数式で証明する。

今,$\Delta \xi$,$\Delta X$をそれぞれ特徴量ベクトルと関節角(ロボットの位置)の微小変動とする。
本章で述べるところの感度$\beta$は$\Delta X$の変動分$v$に対する$\Delta \xi$の変動の下界であると定義する。
即ち,
\begin{equation}
	\beta=\inf\{ \| \Delta\xi \|_2:\| \Delta X \|_2=1 \}
	\label{eq:beta_def}
\end{equation}

今,目標値付近において固定したイメージヤコビアン$\bm{J}_d$を用いることで
これらの変数の関係は次のように示せる。
\begin{equation}
	\Delta \xi = J_d \Delta X = J_d v
	\label{eq:Jacob}
\end{equation}

この関係式において先ほどの感度$\beta$は\shiki{beta_jacob}のように表される。
\begin{equation}
	\beta = \sigma_{min}(\bm{J}_d)
	\label{eq:beta_jacob}
\end{equation}
ここで$\sigma_{min}(\bm{J}_d)$は$\bm{J}_d$の最小特異値である。
すなわち感度$\beta$は実空間での動き$v$に対して特徴量$\xi$が最も動きにくい場合のゲインの下限を示す値である。

感度が低い方向の動きに対しては,画像内でそれを検知することは難しく,またその方向へとロボットを駆動するにはより大きな画像内での指令値を必要とする問題がある。

\subsection{冗長な特徴点の使用による感度の上昇}
新たに特徴ベクトル$\xi$の数を増やすことによって感度$\beta$を上げられることが以下のように証明できる。

先ほどの\shiki{Jacob}において特徴点の個数が$n$の時のイメージヤコビアン$J_{n}$とすると,
特徴点の個数を増やして冗長性を増した$J_{n+1}$は$J_{n}$に新たな行を$J^{(n+1)}$を加えた次のような形になる。
\begin{align}
	J_{n+1} = \left[
	\begin{matrix}
		J_n \\
		J^{(n+1)}
	\end{matrix}
	\right]
\end{align}
任意のベクトル$\bm{w}$に対して次の不等式が成り立つのは自明であるため,
\begin{equation}
	\| \bm{J}_{n+1}\bm{w} \|_2 \geq \| \bm{J}_n\bm{w} \|_2
\end{equation}
$\bm{w}$ に対して $\bm{v}$を代入した場合,\shiki{beta_def}から
\begin{equation}
	\sigma_{min}(\bm{J}_{n+1}) \geq \sigma_{min}(\bm{J}_n)
	\label{eq:beta_grow}
\end{equation}
が導ける。ここで,\shiki{beta_grow}において等式が成り立つのは新たに追加した特徴点がカメラの3次元における変位$\Delta X =v$に全く反応しない時,即ち$J^{(n+1)} v=0$の時である。

従って,新たに追加した特徴点がカメラの運動$\Delta X =v$に反応する時,感度$\beta$は常に増加する。
この時,n個の特徴点とn+1個の特徴点が同じ偏差を与えるならばロボットの空間での偏差はn+1個の特徴点を用いた方が必ず小さくなる。

\section{冗長特徴点を用いた際の精度の向上に関する考察}
複数の特徴点を用いることによって感度,すなわちイメージヤコビアンの最小特異値が上昇することが式によって示された。
また,最適制御の制御則を用いた実験によっても収束速度の向上が見られるが,精度の観点からは検証されていない。

本研究では特徴点を冗長に取ることによる精度の向上を幾つかのこれの結果をシミュレーションで模擬することを考える,


\subsubsection{ピンホールカメラの透視変換モデルに基づく量子化誤差のモデリング}
実際の実験環境下においては特徴点の位置を変化させる様々な誤差が考えうる。
例えば,カメラのレンズの歪みなどにより光軸中心から遠いほど生じる非線形な誤差が生じることはよく知られている\cite{Corke2011}。

また,レンズの歪みを補正したとしても撮像素子上に透過する光量はピクセル値で離散化されるため,たとえ何らかの線形補間などを行ったとしても一定量の量子化誤差が生じる。
SIFTやSURFを始めとした輝度値の微分を用いた特徴点の抽出手法は関数フィッティングを行うことでサブピクセル精度で特徴点の位置を取得可能である\cite{Lowe2004}であるが,
依然として量子化誤差の問題は存在する。

その他にも注視点が実際には厳密には点ではなくある面積をもった空間だった場合に生じる特徴点の座標における「あそび」なども特徴量の誤差として考えうる。

本研究ではそのうち,量子化誤差に着目して量子化誤差の存在下でビジュアルサーボの性能の変化を議論する。

%本章ではカメラの透視変換とそれに基づく分解能に起因する量子化誤差とランダムな位置の変動の2つをモデル化し,シミュレーションにより特徴点の個数に依る,誤差の収束の様子を観測する。



\zu{cameramodel}に示すピンホールカメラの撮像モデルから,あるカメラ空間上の点$(x,y,z)$に存在する物体が焦点距離$f$のカメラ素子に映し出される点の座標$(\xi_x,\xi_y)$は次のような関係で示され,これを透視変換と呼ぶ。
\begin{eqnarray}
	\left(
	\begin{array}{c}
		\xi_x\\
		\xi_y
	\end{array}
	\right)
	=\left(
	\begin{array}{ccc}
		f\dfrac{x}{z}\\[1.5ex]
		f\dfrac{y}{z}
	\end{array}
	\right)
	\label{eq:透視変換}
\end{eqnarray}

ただし,$f$はカメラの焦点距離,$z_0$は基準画像においてカメラと撮像対象との間の距離を表す定数である。

\begin{figure}[t]
	\includegraphics[width=0.8\textwidth]{cameramodel.eps}
	\centering
	\caption{カメラの透視変換}
	\label{fig:cameramodel}
\end{figure}

これは理想的なピンホールカメラの場合のみ成り立ち,より正確に扱う場合はキャリブレーションして求めたカメラの内部パラメータを用いる必要があるが,本章では単純化のためにこのモデルを用いる。


カメラの分解能に起因する量子化誤差を計算した特徴点の座標値を適宜丸めることで模擬する。


%画像ノイズや特徴領域のあそびによる誤差は座標にある振幅を持ったガウシアンノイズを載せることによって模擬する。



\subsection{シミュレーションに基づく最終位置決め誤差の変動の評価}
特徴点を用いた4自由度のイメージベースビジュアルサーボにおいて,先程の量子化誤差を与えた際に生じる誤差と冗長特徴点を用いた際の最終位置決め誤差の検証を行った。
プログラムはMATLAB上で作成し,量子化誤差はMATLABの丸め関数roundを用いて最も近い変数に対して発生させた。

この時,ロボットの4自由度の動きと特徴点座標の関係を表すイメージヤコビアンは
\begin{eqnarray}
	\bm{J}_{i}=
	\left[
	\begin{array}{cccc}
		-\dfrac{f}{Z_i} & 0 & \dfrac{x_i}{Z_i} & -y_i\\[1.5ex]
		0 & -\dfrac{f}{Z_i}  & \dfrac{y_i}{Z_i} & x_i
	\end{array}
	\right]
	\label{eq:jacobian2}
\end{eqnarray}	
を特徴点の数$m$個だけ縦に並べた$2m \times 4$の行列になっている。


シミュレーションにおいて,最初に空間上に特徴点をランダムに配置し,目標位置における特徴点の座標の目標値を取得,その後に指定された数の特徴点を使用してビジュアルサーボを行う。
その際,変動させるパラメータは使用する特徴点の数と量子化誤差計算における有効な桁数の2つである。


はじめに,量子化誤差の模擬において,特徴点座標を整数値で丸めた場合について特徴点の個数による時間応答と最終位置決め精度を評価する。
評価値として位置指令に対して応答が静定した後の最終位置決め誤差の2ノルム$E =\Sigma \bm{e}_i^2 $ を用いており,3次元位置偏差の2ノルムを3D Error,2次元特徴量偏差の2ノルムを2D Errorとする。

また,シミュレーション条件を\tb{冗長}に示す。特徴点の点群は表に示すある3次元座標を中心にランダムに一様になるように配置する。

\begin{table}[!t]
	\begin{center}
		\begin{tabular}{c|c} \hline \hline
			Sampling time& 10ms\\ \hline
			Reference pose $(X_0,Y_0,Z_0,\Theta_0)$ & (0,0,0,0) \\ \hline
			Initial pose $(X_1,Y_1,Z_1,\Theta_1)$ & (10,-10,5,-5) \\\hline
			Focal length $f$ & 10 \\\hline
			Center of points cloud $(X,Y,Z)$ & (0,0,30) \\\hline
			Feedback Gain$(G_X,G_Y,G_Z,G_\Theta)$ & $(10,10,10,5)$\\ \hline \hline
		\end{tabular}
		\caption{シミュレーションにおける諸変数}
		\label{tb:冗長}
	\end{center}
\end{table}


\zu{hikaku2}に異なる特徴点数$NP$を用いた際,それぞれの特徴点数において位置偏差が時間とともに収束する様子を示す。

量子化誤差が存在するため偏差は0へと収束せず,一定の偏差をもって静定する。
\zu{hikaku2}の比較より制御に用いる特徴点の個数を増やしていくほど位置偏差は減少していくことがわかる。
これはすなわち,分解能の低い撮像素子を用いた場合においても,制御に用いる点を冗長に取ることによって分解能の高い撮像素子と同じ結果を得ることができることを示している。

このことは,\zu{AC0_NP2to30}に示すような横軸に特徴点の数,縦軸に最終位置偏差をプロットしたものからも読み解くことができる。
偏差がところどころ上下し,点の増加に伴い単調減少していかない理由は最終位置決め方向へのベクトルのイメージヤコビアン行列の特異ベクトルの値にばらつきがあるためと推測される。
この結果はすなわち,先の章で述べた感度の指標は個々の位置制御のケースに対して必ずしも偏差の減少を一意に対応付けるものではないことを示している。


最後に,特徴点の配置を変えて同様のシミュレーションを400回繰り返した場合の特徴点数と偏差の平均的動向を\zu{beta_Xonly_AV_0}と\zu{beta_Xonly_AV_1_new}に示す。
複数のケースを平均した場合には特徴点の数を増加させた場合に偏差が単調減少していく傾向をもつことが読み取れる。



以上の結果より,特徴点の数を増やすことによって必ずしも希望する位置への収束偏差を減らすことが可能であるわけではないが,平均的には偏差を減少させうるということが示された。

\begin{figure}[!p]
	\centering
	\subfigure[AC=0,NP=2の時の時間応答]{
		\scalebox{0.500}{
			\includegraphics{AC0_NP2.eps}}
		\label{fig:AC0_NP2}}
	\subfigure[AC=0,NP=3の時の時間応答]{
		\scalebox{0.500}{
			\includegraphics{AC0_NP3.eps}}
		\label{fig:AC0_NP3}}
	\subfigure[AC=0,NP=4の時の時間応答]{
		\scalebox{0.500}{
			\includegraphics{AC0_NP4.eps}}
		\label{fig:AC0_NP4}}
	\subfigure[AC=0,NP=5の時の時間応答]{
		\scalebox{0.500}{
			\includegraphics{AC0_NP5.eps}}
		\label{fig:AC0_NP5}}
	\subfigure[AC=0,NP=10の時の時間応答]{
		\scalebox{0.500}{
			\includegraphics{AC0_NP10.eps}}
		\label{fig:AC0_NP10}}
	\subfigure[AC=0,NP=20の時の時間応答]{
		\scalebox{0.500}{
			\includegraphics{AC0_NP20.eps}}
		\label{fig:AC0_NP620}}
	\caption{制御に冗長な特徴点の座標を入力することによって徐々に最終位置決め精度が向上していく様子}
	\label{fig:hikaku2}
	%	\includegraphics[width=0.45\textwidth]{AC1_NP2to30.eps}\label{fig:AC1_NP2to30}
	%	\caption{小数以下丸め桁AC=1の時の,特徴点の増加に伴う感度$\beta$の上昇と位置決め誤差の低減}
\end{figure}


\begin{figure}[t]
	\centering
	\includegraphics[width=0.6\textwidth]{AC0_NP2to30.eps}\label{fig:AC0_NP2to30}
	\caption{小数以下丸め桁AC=0の時の,特徴点の増加に伴う感度$\beta$の上昇と位置決め誤差の低減}
\end{figure}

\begin{figure}[t]
	\centering
	\includegraphics[width=0.6\textwidth]{beta_Xonly_AV_0.eps}\label{fig:beta_Xonly_AV_0}
	\caption{小数以下丸め桁AC=0の時の,特徴点の増加に伴う感度$\beta$の上昇と位置決め誤差の低減の平均的動向}
	\includegraphics[width=0.6\textwidth]{beta_Xonly_AV_1_new.eps}\label{fig:beta_Xonly_AV_1_new}
	\caption{小数以下丸め桁AC=1の時の,特徴点の増加に伴う感度$\beta$の上昇と位置決め誤差の低減の平均的動向}
\end{figure}	



\section{冗長な特徴点によるガウシアン誤差の低減}
量子化誤差と同様に検出した特徴点の座標に対してガウシアン誤差が存在したときのシミュレーションを行った。
ガウシアン誤差は全ての特徴点の位置と時間経過に対して独立で,分散は固定されているとする。

この際も同様に時系列で生じるノイズに対して,冗長に特徴点を取ることでこれを低減することが出来ることを示した。
\begin{figure}[!t]
	\centering
	\includegraphics[width=10cm]{gauss_kekka.eps}
	\caption{ガウシアンノイズを加えた場合の位置の時間波形}
	\includegraphics[width=10cm]{For_gaussian_noise.eps}
	\caption{特徴点の数を増やした時の収束精度}
\end{figure}

\begin{figure}[!t]
	\centering
	\includegraphics[width=10cm]{For_gaussian_noise_ave20_val1e-1.eps}
	\caption{特徴点の数を増やした時の収束精度を20回平均したもの(分散0.1)}
	\includegraphics[width=10cm]{For_gaussian_noise_ave20_val2e-1.eps}
	\caption{特徴点の数を増やした時の収束精度を20回平均したもの(分散0.2)}
\end{figure}

\section{特徴点の個数に関する簡易的な考察}
これらの結果の考察において,
平均的な動向を確認すると\zu{ac0fitted2d}や\zu{ac1fitted2d}に示すとおり,特徴点の個数$n$に対して,反比例の形$\frac{a}{n}+b$で最終位置決め誤差をうまくフィッテイングすることができるように思われる。

統計学において,大数の法則から計測点を増やすことで誤差の分布を縮小することが可能であり,今回のシミュレーションの結果である制御に用いる点を増加させることで位置決め誤差を低減するモデルは非常に似通っていると思われる。
今後の課題として,計算量等のペナルティを課すことによって最適な特徴点の個数を決定づける方法が考えうる。
その上で特徴点の取得する位置によっても前章で定義した感度が変化することが知られており,その解析に基づく特徴点の選択の手法を提案することが可能と考える。


\begin{figure}[t]
	\centering
	\includegraphics[width=0.7\linewidth]{fig/AC0fitted2D}
	\caption{\fig{AC0_NP2to30}における特徴点の個数と最終位置決め誤差の平均的動向を反比例の関数$\frac{a}{n}+b$でフィッティングしたもの。}
	\label{fig:ac0fitted2d}
\end{figure}


\begin{figure}[t]
	\centering
	\includegraphics[width=0.7\linewidth]{fig/AC1fitted2D}
	\caption{\fig{AC1_NP2to30}における特徴点の個数と最終位置決め誤差の平均的動向を反比例の関数$\frac{a}{n}+b$でフィッティングしたもの。}
	\label{fig:ac0fitted2d}
\end{figure}


\begin{figure}[t]
	\centering
	\includegraphics[width=0.7\linewidth]{Gaussian1_2d}
	\caption{ガウシアン誤差を加えた場合の特徴点の数と最終位置決め誤差の平均的動向を反比例の関数でフィッティングしたもの}
	\label{fig:gaussian12d}
\end{figure}

%\subsection{実験に基づく検証}

%\section{特徴点の画像内の位置に関する考察}
%特徴点の数を増やすことによってイメージヤコビアンの最小特異値が増加し,結果として動きの精度が向上することは確認できた。
%続いて,特徴点の位置と配置に関して考察を行った。
%
%人間の感覚的には,全体としてバランスの良い配置が望ましく,また画像の中心点に点が寄るほど,距離方向への変位に対して鈍感になること等が予想される。
%
%
%
%1つ1つの点に対応する
%\shiki{Jacob基本}
%%\shiki{Jacob基本}
%に示すイメージヤコビアンについて計算すると行列のRankが
%動きの自由度に対して足りずに意味の抽出しづらい結果になるので
%Rankを動きの自由度である6以上にするように必要最低限の点を事前に配置した上で,次に追加する点を選び,その点を追加したことで感度がどれほど向上するかをもってこれを評価した。


\end{document}
\include{chapter/PhaseCorrelation}
\include{chapter/POCbasedVisualServo}
\include{chapter/VideoTracking}
\documentclass[12pt,a4paper,oneside,onecolumn,fleqn,dvipdfmx]{jreport}

\usepackage{../main/preamble}

\begin{document}

    \chapter{結論}
        ここに結論を書く。

\end{document}


\newpage
\phantomsection
\addcontentsline{toc}{chapter}{謝辞}
\documentclass[fleqn]{jreport}
\usepackage[dvipdfmx, bookmarkstype=toc, colorlinks=false, pdfborder={0 0 0}, bookmarks=true, bookmarksnumbered=true]{hyperref}
\usepackage{pxjahyper}
\usepackage{docmute}

\usepackage{booktabs}
\usepackage[dvipdfmx]{graphicx,color}
%\usepackage[dvipdfmx]{color}
\usepackage{amsmath}
\usepackage{ascmac}
\usepackage{times}

\usepackage{epsf}
\usepackage{verbatim}
\usepackage{subfigure}
\usepackage[varg]{txfonts}

\usepackage{multicol}
\usepackage{pifont}
%\usepackage{docmute} %単体でコンパイルできるようにする
\usepackage{mathtools} % dcases を使う

\renewcommand{\bf}{\bfseries}
\renewcommand{\gt}{\gtfamily}
\renewcommand{\sf}{\sffamily}
\label{alg1}

%\usepackage{algorithm}
%\usepackage{algorithmic}
%\newcommand{\AND}{\algorithmicand{} }
%\renewcommand{\algorithmicrequire}{\textbf{Input:}}
%\renewcommand{\algorithmicensure}{\textbf{Output:}}

\newcommand{\bm}[1]{\mbox{\boldmath $#1$}}
\newcommand{\eq}[1]{(\ref{eq:#1})}
\newcommand{\zu}[1]{\fig{#1}}
\newcommand{\shiki}[1]{式(\ref{eq:#1})}
\newcommand{\fig}[1]{Fig.\ \ref{fig:#1}}
\newcommand{\grp}[1]{Fig.\ \ref{grp:#1}}
\newcommand{\tb}[1]{Tab.\ \ref{tb:#1}}
\newcommand{\defeq}{:=}
\newcommand{\bvec}[1]{\mbox{\boldmath $#1$}} %太字
\newcommand{\tvec}[1]{\bvec{#1}^{\mathsf{T}}} %太字転置

\newtheorem{theorem}{Theorem}
\newtheorem{definition}[theorem]{Definition}
\newtheorem{lemma}[theorem]{Lemma}
\newtheorem{corollary}[theorem]{Corollary}

\graphicspath{{../fig/}}



\begin{document}
\chapter*{謝辞}
本研究を進めるにあたり,毎週の研究相談会及び研究発表会の場において,熱意ある指導と適切な助言をしてくださった藤本博志准教授,研究発表会の場において丁寧な指導をしてくださった堀洋一教授に心から感謝いたします。

実験装置のセットアップにおいて多大な尽力を頂いた当研究室卒業生の郡司様,犬飼様,
直接赴いて手助けをしてくださった梅津様,富田様をはじめとする安川電機の方々に深く感謝いたします。

ナノチームの博士課程の大西様,山田様,矢崎様を始めとする先輩方には論文の校正や書き方の相談を始め,面倒を見てくださり深く感謝しています。
また,同チームの延命様,下田様,長谷川様を始めとする研究室の同期や後輩達には,日頃から研究に関する議論や相談をしていただき,学業の面でも精神面でも大変支えになりました。

誠にありがとうございました。


\end{document}
\newpage
\phantomsection
\addcontentsline{toc}{chapter}{参考文献}
%\bibliographystyle{ieeetran}
\bibliographystyle{tieice}
\bibliography{ref}
\newpage
\phantomsection
\addcontentsline{toc}{chapter}{発表文献}
\documentclass[fleqn]{jreport}

\begin{document}
\chapter*{発表文献}

\section*{査読付国際会議論文}
\noindent
\begin{tabular}{ccl}
	[1]&著\hspace{2em}者&\underline{Y. Ri}, H. Fujimoto\\
	&題\hspace{2em}名&Proposal of Visual Servoing using Phase-Only-Correlation (POC) \\
	&会\hspace{0.5em}議\hspace{0.5em}名&  The 41th annual conference of the IEEE Industrial Electronics Society (IECON2015) \\
	&場\hspace{2em}所&Yokohama, Japan\\
	&発\hspace{0.5em}表\hspace{0.5em}日& 10th, November, 2015\\
	&開催期間\hspace{0.5em}& 9th--12th, November, 2015 \\
\end{tabular}\\
\\
\\
\begin{tabular}{ccl}
	[2]&著\hspace{2em}者&\underline{Y. Ri}, H. Fujimoto\\
	&題\hspace{2em}名&Image Based Visual Servo Application on Video Tracking with
	Monocular Camera \\
	&&Based on Phase Correlation Method\\
	&会\hspace{0.5em}議\hspace{0.5em}名& 2017 SAMCON\\
	&&The 3rd IEEJ International Workshop on Sensing, Actuation, Motion Control, and Optimization \\
	&場\hspace{2em}所&Niigata, Japan\\
	&発\hspace{0.5em}表\hspace{0.5em}日& 4th, March, 2017\\
	&開催期間\hspace{0.5em}& 6th--8th, March, 2017 \\
\end{tabular}\\

\section*{国内会議論文}
\noindent
%
\begin{tabular}{ccl}
	[3]&著\hspace{2em}者&\underline{李 尭希},藤本博志\\
	&題\hspace{2em}名&画像の変形パラメータを用いたビジュアルサーボ\\
	&会\hspace{0.5em}議\hspace{0.5em}名&平成25年産業計測制御/メカトロニクス制御合同研究会, \\
	&&IIC-13-101, MEC-13-101, pp.67-72, 2013\\
	&場\hspace{2em}所&千葉大学,千葉県\\
	&発\hspace{0.5em}表\hspace{0.5em}日& 2013年3月8日\\
	&開催期間\hspace{0.5em}& 2013年3月7--8日 \\
\end{tabular}\\
\\
\\
\begin{tabular}{ccl}
	[4]&著\hspace{2em}者&\underline{李 尭希},藤本博志\\
	&題\hspace{2em}名&テンプレートに対する画像の平行移動・回転・拡大縮小変位を特徴量に用いる\\
	&&位相限定相関法に基づくビジュアルサーボの提案\\
	&会\hspace{0.5em}議\hspace{0.5em}名&第33回 日本ロボット学会 学術講演会 1D1-04\\
	&場\hspace{2em}所&東京電機大学,東京\\
	&発\hspace{0.5em}表\hspace{0.5em}日& 2015年9月3日\\
	&開催期間\hspace{0.5em}& 2013年9月3日--5日 \\
\end{tabular}\\
\\
\\
\begin{tabular}{ccl}
	[5]&著\hspace{2em}者&\underline{李 尭希},藤本博志\\
	&題\hspace{2em}名&画像の平行移動・回転・拡大縮小量を特徴量に用いるビジュアルサーボと\\
	&&その検出手法に関する基礎検討\\
	&会\hspace{0.5em}議\hspace{0.5em}名&平成27年メカトロニクス制御研究会/精密サーボシステムと制御技術, \\
	&場\hspace{2em}所&電気学会会議室,東京\\
	&発\hspace{0.5em}表\hspace{0.5em}日& 2015年10月9日\\
	&開催期間\hspace{0.5em}& 2015年10月9日 \\
\end{tabular}\\
\\
\\
\begin{tabular}{ccl}
	[6]&著\hspace{2em}者&\underline{李 尭希},藤本博志\\
	&題\hspace{2em}名&単眼カメラのビデオ動画に基づく動作再現を目指した\\
	&&ビジュアルサーボの応用に関する研究\\
	&会\hspace{0.5em}議\hspace{0.5em}名&第34回ロボット学会学術講演会 \\
	&場\hspace{2em}所&山形大学小白川キャンパス,山形\\
	&発\hspace{0.5em}表\hspace{0.5em}日& 2016年9月7日\\
	&開催期間\hspace{0.5em}& 2016年9月7日--9日 \\
\end{tabular}\\

\section*{投稿準備中の論文誌}
\noindent
\begin{tabular}{ccl}
	[7]&著\hspace{2em}者&\underline{李 尭希},藤本博志\\
	&題\hspace{2em}名&位相限定相関法を用いたビジュアルサーボ \\
	&論文誌名\hspace{0.5em}&日本ロボット学会誌\\
\end{tabular}

%\section*{掲載済みの論文誌}
%\noindent
%\begin{tabular}{ccl}
%[1]&著\hspace{2em}者&\underline{なまえ},共著者\\
%&題\hspace{2em}名&論文題目\\
%&論文誌名\hspace{0.5em}&電気学会論文誌D, vol. 134--D, no. 3, pp. 293--300 (2014)\\
%\end{tabular}\\
%
%
%
%\section*{受\hspace{.5zw}賞}
%\noindent
%\begin{tabular}{ccl}
%[11]&受\hspace{0.5em}賞\hspace{0.5em}者&\underline{なまえ}\\
%&題\hspace{2em}名&論文題目 \\
%&受\hspace{0.5em}賞\hspace{0.5em}名& 賞の名前 \\
%&受\hspace{0.5em}賞\hspace{0.5em}日&日時\\
%\end{tabular}\\

\end{document}

%\gakkai         %ここから発表文献
%\include{chapter/10_Present}
%
%
\appendix       %ここから付録
\newpage
\phantomsection
\addcontentsline{toc}{chapter}{付録}
\documentclass[fleqn]{jreport}
\usepackage[dvipdfmx, bookmarkstype=toc, colorlinks=false, pdfborder={0 0 0}, bookmarks=true, bookmarksnumbered=true]{hyperref}
\usepackage{pxjahyper}
\usepackage{docmute}

\usepackage{booktabs}
\usepackage[dvipdfmx]{graphicx,color}
%\usepackage[dvipdfmx]{color}
\usepackage{amsmath}
\usepackage{ascmac}
\usepackage{times}

\usepackage{epsf}
\usepackage{verbatim}
\usepackage{subfigure}
\usepackage[varg]{txfonts}

\usepackage{multicol}
\usepackage{pifont}
%\usepackage{docmute} %単体でコンパイルできるようにする
\usepackage{mathtools} % dcases を使う

\renewcommand{\bf}{\bfseries}
\renewcommand{\gt}{\gtfamily}
\renewcommand{\sf}{\sffamily}
\label{alg1}

%\usepackage{algorithm}
%\usepackage{algorithmic}
%\newcommand{\AND}{\algorithmicand{} }
%\renewcommand{\algorithmicrequire}{\textbf{Input:}}
%\renewcommand{\algorithmicensure}{\textbf{Output:}}

\newcommand{\bm}[1]{\mbox{\boldmath $#1$}}
\newcommand{\eq}[1]{(\ref{eq:#1})}
\newcommand{\zu}[1]{\fig{#1}}
\newcommand{\shiki}[1]{式(\ref{eq:#1})}
\newcommand{\fig}[1]{Fig.\ \ref{fig:#1}}
\newcommand{\grp}[1]{Fig.\ \ref{grp:#1}}
\newcommand{\tb}[1]{Tab.\ \ref{tb:#1}}
\newcommand{\defeq}{:=}
\newcommand{\bvec}[1]{\mbox{\boldmath $#1$}} %太字
\newcommand{\tvec}[1]{\bvec{#1}^{\mathsf{T}}} %太字転置

\newtheorem{theorem}{Theorem}
\newtheorem{definition}[theorem]{Definition}
\newtheorem{lemma}[theorem]{Lemma}
\newtheorem{corollary}[theorem]{Corollary}

\graphicspath{{../fig/}}



\begin{document}
\chapter{位相相関の手法の拡張}
OjansivuらによるBIPC(Blur-Invariant Phase Correlation)という手法ではフーリエ変換の畳み込みの性質を用いて,
モーションブラーによる画像変位の影響を取り除く手法が紹介されている。


画像$a$にブラーとノイズがかかった場合の画像を$b$とすると以下の数式に示す対応関係が生じる。
\begin{align}
b(n_1,n_2) = a(n_1,n_2) * h(n_1,n_2) + n(n_1,n_2)
\end{align}
ここで$h(n_1,n_2)$は畳み込まれるブラー成分であり$n(n_1,n_2)$はノイズ成分である。
ノイズが十分小さいとしてこれを2次元フーリエ変換すると畳込みとフーリエ変換の性質より,畳込みは周波数領域での乗算で次のように表される。
\begin{align}
B(k_1,k_2) = A(k_1,k_2) \dot  H(k_1,k_2)
\end{align}
ここで,$B$を振幅で正規化した位相成分に着目すると,
\begin{align}
\frac{B(k_1,k_2)}{|B(k_1,k_2)|} = \exp{-i\phi_b(k_1,k_2)}= \exp{-i(\phi_a(k_1,k_2) +\phi_h(k_1,k_2))}
\end{align}
のように表すことができる。


ところで,ブラーの変形量は中心対称とすると,$\phi_h(k_1,k_2)$の値は$0$か$\pi$に限定される。
この仮定のもと,自然数$n$を用いた位相成分の$2n$乗を考えると,
\begin{align}
\left( \frac{B(k_1,k_2)}{|B(k_1,k_2)|} \right)^{2n} =  \exp{-i(2n\phi_a(k_1,k_2) +2n\phi_h(k_1,k_2))} = \exp{-i (2n\phi_a(k_1,k_2))}
\end{align}
の様に書けるためブラーの成分の影響を無視することができることがわかる。

従って画像入力$f$と$g$に対して新たにPhaseCorrelation関数$S(k_1,k_2)$を
\begin{align}
S(k_1,k_2) &= \left( \frac{F(k_1,k_2)\bar{G(k_1,k_2)}}{|F(k_1,k_2)||G(k_1,k_2)|} \right)^{2n}
\end{align}
と置くことで,これをフーリエ逆変換した$s(n_1,n_2)$は次のように表せる。
\begin{align}
s(n_1,n_2) = \delta(n_1-2nx,n_2-2ny)
\end{align}
ここで,$f$と$g$の間の平行移動量を$(x,y)$と置いており,$\delta$はディラックのデルタ関数である。
一般には$n=1$とするのが望ましい。

この手法を用いることで有効に画像の変異を抽出できる範囲が半分になることに留意する必要がある。

\chapter{特徴点座標群が与えられた場合のAffineまたはhomography変換行列の推定手法}

\section{変換行列の定義}
変換前の画像の座標群を$(x^*_i,y^*_i)^T$,変換後の座標群を$(x_i,y_i)^T$と定義$(i=1,2,3...)$する。
ここで,Affine変換は一般に以下のように定義される。
\begin{align}
\begin{pmatrix}
x_i \\ 
y_i
\end{pmatrix} =\begin{pmatrix}
a_1 & a_3 & a_5 \\ 
a_2 & a_4 & a_6
\end{pmatrix} \begin{pmatrix}
x_i^* \\ 
y_i^* \\ 
1
\end{pmatrix}  \label{eq:Affine変換行列}
\end{align}
$a_{i} \ (i=1,2,3,4,5,6)$で表される行列はAffine変換行列と呼ばれ,$a_{1}~a_{4}$は回転と各軸方向における拡大縮小を表し,$a_5,a_6$は平行移動量を表す。
同様にして,Homography変換も以下のような形で書ける。
\begin{align}
\begin{pmatrix}
x_i \\ 
y_i \\
1
\end{pmatrix} =\begin{pmatrix}
g_1 & g_4 & g_7 \\ 
g_2 & g_5 & g_8 \\ 
g_3 & g_6 & g_9
\end{pmatrix}  \begin{pmatrix}
x_i^* \\ 
y_i^* \\ 
1
\end{pmatrix}  
\end{align}
$g_1~g_9$で表される行列はHomography変換行列と呼ばれ,一般には$g_9=1$としてスケールをあわせる場合が多い。その場合,適切なスケーリングファクター$s$を用いて次のように表されることが多い。
\begin{align}
\begin{pmatrix}
x_i \\ 
y_i \\
1
\end{pmatrix} =s \begin{pmatrix}
g_1 & g_4 & g_7 \\ 
g_2 & g_5 & g_8 \\ 
g_3 & g_6 & 1
\end{pmatrix}  \begin{pmatrix}
x_i^* \\ 
y_i^* \\ 
1
\end{pmatrix}  \\
\mbox{while,}
s=\frac{1}{g_{3}x^*_i+g_{6}y^*_i+1}
\label{eq:Gscale}
\end{align}
$g_3=g_6=0$とおいた場合,これはAffine変換行列と同義になる。Homography行列を用いることによってAffine変換で扱われる回転と平行移動,軸方向への拡大縮小の他に,任意の2次元画像上の変換を記述することができる。


\section{変換行列の同定法}
\subsection{Affine行列の推定}
Affine行列を推定する際は一般に線形解法が用いられる。
1組の対応点を取得することにより\shiki{AffineEquation}2つの方程式を得ることができるため\shiki{Affine変換行列}に示すAffine行列の全パラメータを求めるためには最低でも3つの対応点を探す必要がある。
\begin{align}
\left(
\begin{array}{c}
x_i\\
y_i
\end{array}
\right)=\left(
\begin{array}{ccc}
\vspace{2mm}
a_{1}x^*_i+a_{3}y^*_i+a_{5} \\ 
a_{2}x^*_i+a_{4}y^*_i+a_{6}
\end{array}
\right) \label{eq:AffineEquation}
\end{align}
また,この\shiki{AffineEquation}を行列の各成分を要素に持つベクトル対する係数行列の形に書き直すと,次のようになる。
\begin{align}
\begin{pmatrix}
x^*_i & 0 & y^*_i & 0 & 1 & 0 \\ 
0 & x^*_i & 0 & y^*_i & 0 & 1
\end{pmatrix} \begin{pmatrix}
a_1 \\ 
a_2 \\ 
a_3 \\ 
a_4 \\ 
a_5 \\ 
a_6
\end{pmatrix} = \begin{pmatrix}
x_i \\ 
y_i
\end{pmatrix} \label{eq:affineEq2}
\end{align}
したがってこれを対応付けた$n (n \geq 3) $個の点について行列を縦に並べることで,線形方程式の形にすることができる。
\begin{align}
&\bm{Ax}=\bm{b}\label{eq:affineLinear}  \\
&\bm{A} = \begin{pmatrix}
x^*_1 & 0 & y^*_1 & 0 & 1 & 0 \\ 
0 & x^*_1 & 0 & y^*_1 & 0 & 1 \\
& & \vdots &   &\\
& & \vdots &   &\\
x^*_n & 0 & y^*_n & 0 & 1 & 0 \\ 
0 & x^*_n & 0 & y^*_n & 0 & 1
\end{pmatrix} , \hspace{2mm}
\bm{x} = \begin{pmatrix}
a_1 \\ 
a_2 \\ 
a_3 \\ 
a_4 \\ 
a_5 \\ 
a_6
\end{pmatrix} , \hspace{2mm}
\bm{b}= \begin{pmatrix}
x_1 \\ 
y_1 \\
\vdots \\
\vdots \\
x_n \\ 
y_n
\end{pmatrix} \nonumber 
\end{align}
\shiki{affineLinear}は$\bm{A}$の擬似逆行列$\bm{A}^+$を計算することで,最小二乗解を次のように求めることができる。
\begin{align}
\bm{x} = \bm{A}^+\bm{b}
\end{align}

\subsection{Homography行列の推定}
ここではHomography行列の線形解法について述べる。
Homography行列の場合,対応する点1つにつき次のような関係式があるため,計4つの特徴点を用いることでHomographyを計算できる。
\begin{align}
\left(
\begin{array}{c}
x_i\\
y_i 
\end{array}
\right)=\left(
\begin{array}{ccc}
\vspace{2mm}
\dfrac{g_{1}x^*_i+g_{4}y^*_i+g_{7}}{g_{3}x^*_i+g_{6}y^*_i+1} \\ \vspace{2mm}
\dfrac{g_{2}x^*_i+g_{5}y^*_i+g_{8}}{g_{3}x^*_i+g_{6}y^*_i+1} 
\end{array}
\right)\label{eq:HEquation}
\end{align}
\shiki{HEquation}を展開して,\shiki{affineEq2}と同様に行列の要素のベクトル表示に展開すると,次のようになる。
\begin{align}
\begin{pmatrix}
x^*_i & 0 & -x^*_i x_i & y^*_i & 0 & -y^*_i x_i& 1 & 0  \\ 
0 & x^*_i & -x^*_i y_i & 0 & y^*_i & -y^*_i y_i& 0 & 1
\end{pmatrix} \begin{pmatrix}
g_1 \\ 
g_2 \\ 
g_3 \\ 
g_4 \\ 
g_5 \\ 
g_6 \\
g_7 \\
g_8
\end{pmatrix} = \bm{0} \label{eq:HEq2}
\end{align}
この式は\shiki{HLinear}の行列$A$の零ベクトル,近似的には最小特異値に対応するベクトルを求める問題に帰結する。
\begin{align}
&\bm{Ax}=\bm{0}\label{eq:HLinear} \\
&\bm{A}=\begin{pmatrix}
x^*_1 & 0 & -x^*_1 x_1 & y^*_1 & 0 & -y^*_1 x_1& 1 & 0  \\ 
0 & x^*_1 & -x^*_1 y_1 & 0 & y^*_1 & -y^*_1 y_1& 0 & 1 \\
& & & & \vdots & & &\\
& & & & \vdots & & &\\
x^*_n & 0 & -x^*_n x_n & y^*_n & 0 & -y^*_n x_n& 1 & 0  \\ 
0 & x^*_n & -x^*_n y_n & 0 & y^*_n & -y^*_n y_n& 0 & 1
\end{pmatrix},\hspace{5mm}
\bm{x} = \begin{pmatrix}
g_1 \\ 
g_2 \\ 
g_3 \\ 
g_4 \\ 
g_5 \\ 
g_6 \\
g_7 \\
g_8
\end{pmatrix} \nonumber
\end{align}
MATLAB上では固有値分解を行った際の最小固有ベクトルを元に$\bm{x}$を推定する。

\section{RANSACを用いた外れ値の回避}
上記の複数の点を用いた変換パラメータ推定法は外れ値などに大きく影響を受けるため,繰り返し動作によって外れ値を回避する手法が提案されている。
概要は以下の通りである。
\begin{enumerate}
\item 総データ個数がU個あるデータから、ランダムでn個のデータを取り出し,取り出したn個のデータから、パラメータを求める。
\item 求めたパラメータを総データ点数から取り出したn個のデータを除いたものに対してそれぞれ当てはめて誤差を計算し誤差が許容範囲内であれば,パラメータに対して投票を行う。これを残りの全てのデータに対して行う。
\item 手順1と2を何度か繰り返して投票数が一番多かったパラメータをひとまず採用する。
採用したパラメータを使ってすべてのデータに再度式を適用し,誤差が許容範囲内のものを抽出を行う。
\item 抽出したデータを元に,手順1,2を繰り返して再度パラメータ求め直す。
\item パラメータを用いた際の誤差がある範囲に収まったならばそのパラメータを推定値とする。
\end{enumerate}


\end{document}

%
%
%%%%%%%%%%%%%%%%%%%%%%%%%%%%%%%%%%%%%%%%%%%%%%%%%%%%%%%%%%%%%%%%%%%%%%%%%%%%

\end{document} 