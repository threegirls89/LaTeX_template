\documentclass[a4paper,10pt,twocolumn,fleqn]{jarticle}

 %%%%%%%%     プリアンブル     %%%%%%%%
\usepackage{houkoku}

 %%%%%%%%     マージン調整     %%%%%%%%
\setlength{\textwidth}{180mm}
\setlength{\columnsep}{10mm}
\setlength{\textheight}{260mm}
\setlength{\topmargin}{-45mm}
\addtolength{\topmargin}{25mm}
\setlength{\oddsidemargin}{-1in}

\addtolength{\oddsidemargin}{16mm}
\setlength{\evensidemargin}{-1in}
\addtolength{\evensidemargin}{16mm}

%%%%%%%%%%% 報告書番号 %%%%%%%%%%%%%
\newcommand{\Reportnum}[0]{1}%報告書番号

\makeatletter
\def\ps@myplain{%
 \let\@mkboth\@gobbletwo
 \def\@oddhead{ XX-Report-\thepage} %奇数ページヘッダ,名前の前のスペースは消さないように
 \def\@oddfoot{ \thepage \ }%奇数ページフッタ
 \let\@evenhead\@oddhead %偶数ページヘッダ
 \let\@evenfoot\@oddfoot %偶数ページフッタ\@topnewpage
}
\makeatother
\pagestyle{myplain} %ページスタイル


%%%%%%%%%%%%%%%%%%%%%%%%%%%%%%%%%%%%%%%%%%%%%%%%%%%%%%%%%%%%%%%
\begin{document}
%\onecolumn
\twocolumn[
\begin{flushright}
\today\\
東京大学 堀藤本研究室 XX XX
\end{flushright}
\begin{center}
{\LARGE 報告会 (\Reportnum)}
\end{center}
\vspace{2zw}
]

\section{報告内容}
説明。
\begin{enumerate}
	\item 日本語用報告会テンプレです。
	\item 2カラムです。
	\item \begin{verbatim*}
	\Reportnum
	\end{verbatim*}の所の数を変えることで報告会番号を調整します。
	\item XXのところを自分の名前に合わせて整形して見ましょう。
\end{enumerate}


\end{document}
 