\documentclass[11pt,a4paper,oneside,onecolumn,fleqn,dvipdfmx]{jsarticle}

\usepackage{preamble}

\addtolength{\textheight}{\topskip}
\setlength{\fullwidth}{170truemm}
\setlength{\textwidth}{\fullwidth}
\setlength{\textheight}{40\baselineskip}
\setlength{\voffset}{-0.55in}
\setlength{\oddsidemargin}{-0.4truemm}
\setlength{\evensidemargin}{-10.4truemm}

\begin{document}

\thispagestyle{empty}
\begin{center}
    \vspace*{20pt}
    {\fontsize{60pt}{0pt}\selectfont{卒業論文}}
    \vskip 80pt
    {\fontsize{40pt}{0pt}\selectfont{題目}}
    \vskip 80pt
    {\fontsize{30pt}{0pt}\selectfont{yyyy年mm月dd日 \ 提出}}
    \vskip 60pt
    {\fontsize{30pt}{0pt}\selectfont{指導教員}}
    \vskip 10pt
    {\fontsize{30pt}{0pt}\selectfont{堀 洋一 教授}}
    \vskip 10pt
    {\fontsize{30pt}{0pt}\selectfont{藤本 博志 准教授}}
    \vskip 60pt
    {\fontsize{30pt}{0pt}\selectfont{電気電子工学科}}
    \vskip 20pt
    {\fontsize{30pt}{0pt}\selectfont{00-000000\ \ 名前}}
\end{center}
\clearpage

\tableofcontents
\clearpage

\listoffigures
\listoftables
\clearpage

\section{序論}
ここに序論を書く。

\section{実験装置}
ここに実験装置を書く。

\section{モデリング}
ここにモデリングを書く。

\section{従来法}
ここに従来法を書く。

\section{提案法}
ここに提案法を書く。

\section{シミュレーション}
ここにシミュレーションを書く。

\section{実験}
ここに実験を書く。

\section{結論}
ここに結論を書く。

\bibliographystyle{tipsj}
\bibliography{ref}

\end{document}
